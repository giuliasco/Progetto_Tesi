\chapter{Tecnica Ottimizzata}
\label{cap:3}
In questo capitolo viene presentata un'ottimizzazione al problema descritto nel capitolo precedente.
Tale ottimizzazione si basa sul principio delle decomposizioni bilanciate.
Si fa vedere, inoltre, come vengono modificati i precedenti dettagli implementativi ottenendo un miglioramento delle perfomance.

\section{Decomposizioni Bilanciate di un albero}
\label{cap:3 par:1}
Si vuole andare a dimostrare in questa sezione che dato un albero T \`e sempre possibile ricavare una scomposizione bilanciata dell'albero.
\\
Prima di poter enunciare e dimostrare il risultato principale occorre dare delle nozioni preliminari.

\newtheorem{definizione}{Definizione}[section]

\begin{definizione}
	\label{definizioneDeco}
Sia $T_r$ un albero radicato nel nodo r, con k nodi.
Diremo che la coppia (A,B), dove  A e B sono insiemi contenenti i nodi di $T_r$, \`e una decomposizione per l'albero $ T_r $ se:
\begin{itemize}
	\item $| A | + | B | = k$
	\item $A \cap B = \{r\}$.
\end{itemize}
\end{definizione}


\begin{definizione}
\label{lemmaDeco}
Dato un albero $ T $ con $ k $ nodi, diremo che $ (A,B) $ \`e una decomposizione bilanciata se:
\begin{equation*}
	\max{ \{|A| , |B| \} }  \le  f(k)
\end{equation*}
con $ f $ una funzione definita su $ k $.
\end{definizione}





\begin{definizione}
Per ogni nodo $ v $ di un albero $ T $, le diramazioni di $ T $  rispetto a $ v $, sono tutti i sottoalberi massimali di $ T $ non contenenti $ v $. 
Per ogni $ v \in T $, si definisce $\alpha(v)$ come il grado della diramazione di $ v $ con il maggior numero di nodi.\\
Un nodo $ v $ di un albero $ T $ con $ n $ nodi, \`e un nodo centroide se $\alpha(v)\le\frac{n}{2}$.
\end{definizione}\mbox{}

Il centroide di un albero non \`e necessariamente unico, infatti Jordan \cite{jordan1869assemblages}  ha dimostrato che, dato un albero $ T $ con $ n $ nodi:
\begin{enumerate}
	\renewcommand{\labelenumi}{\roman{enumi}}
	\item $ T $ ha un singolo centroide $ v $ e $\alpha(v) < \frac{n}{2}$;
	\item$ T $ ha due nodi centroidi (adiacenti) $v_1$ e $v_2$ tali che $\alpha(v_1) = \alpha(v_2) = \frac{n}{2}$, in questo caso il numero di nodi $ n $ \`e pari.
\end{enumerate}

Esistono diversi algoritmi per la ricerca del centroide, quello utilizzato in questa tesi \`e l'algoritmo di Jordan \cite{jordan1869assemblages} che  ha una complessit\`a temporale lineare nel numero di nodi. \\
Il primo passo da effettuare \`e determinare $\alpha(v)$ per ogni nodo $ v \in T$.\\
Si indica  $\delta(z)$ come il fattore di diramazione di un albero. ossia il numero di nodi incontrati durante una visita DFS (in profondit\`a) effettuata a partire dalla radice $ z $.
Siano $\delta(z_i)$ i fattori di bilanciamento ottenuti da tutte le possibili diramazioni $ i $ di $ T $ unite a $ v $ che non lo contengono, allora, $ \alpha(v) = \max \{\delta(z_i)\} $.

Una volta calcolato il valore di $ \alpha(v) $ $ \forall v \in T $ si verifica per quali valori  risulta $\alpha(v)\le\frac{n}{2}$.\\
Nel caso ci fosse un unico nodo $ v $ che soddisfa la precedente espressione, come in (i),  allora tale nodo rappresenta l'unico  centroide dell'albero $ T $.
Nel caso, invece, ce ne fossero due, come definito in (ii), per esempio $ v_1 $ e $ v_2 $,  l'albero $ T $ conterr\`a due centroidi, rispettivamente $ v_1 $ e $ v_2 $.\\
Nell'esempio \ref{es1} si pu\`o vedere l'applicazione dell'algoritmo per la ricerca del centroide.
	\begin{figure}[htbp]
		\centering
		\includegraphics[width=5cm]{capitolo3/grafo2}
		\caption{Albero $ T $  per la ricerca del centroide} 
		\label{fig:2}
\end{figure}
\mbox{}\\

\newtheorem{esempio}[definizione]{Esempio}
\begin{esempio}
	\label{es1}
Si consideri l'albero T in figura \ref{fig:2} per la ricerca del nodo centroide.
Per ogni nodo $ v $ di $ T $ numerato da 0 a 10,  viene calcolato $\alpha(v)$ . \\
Quello che si ottiene \`e :


\begin{center}
	\begin{tabular}{ c c c c c  }
		$\alpha(0) = 6$ & & $\alpha(1) = 7$ & & $\alpha(2) = 5$ \\ 
		$\alpha(3) = 9$ && $\alpha(4) = 10$ &&  $\alpha(5) =  7$ \\  
		$\alpha(6) = 10$ && $\alpha(7) = 10$ && $\alpha(8) = 10$ \\
		$\alpha(9) = 10$ && $\alpha(10) = 10$ &&
	\end{tabular}
\end{center}

Poich\'e $ \left\lfloor\frac{n}{2} \right\rfloor = \left\lfloor \frac{11}{2} \right\rfloor = 5$, l'unico nodo per cui la disuguaglianza risulta vera \`e il nodo 2, infatti $5\le 5$.\\
Poich\`e il numero di nodi \`e dispari certamente questo sar\`a l'unico centroide dell'albero T (figura \ref{fig:2}). 
\demo
\end{esempio}\mbox{}\\

L'ultimo punto da considerare prima di poter enunciare e dimostrare il risultato principale di questa sezione riguarda la definizione di un algoritmo valido per comporre due insiemi di nodi, che chiameremo $ T' $ e $ T'' $, in maniera $ f(k) $-bilanciata.\\
Siano dati in input un albero $ T $, con $ k $ nodi, ed un fattore di bilanciamento definito da una funzione $ f(k) $, poich\'e non considero la radice di $ T $, si avr\'a che $ f(k) = f(k-1) $.
Si suppone inoltre, senza perdita di generalit\`a, che i sottoalberi radicati nei figli della radice $ r $ di $ T $ siano ordinati con ordine non crescente.
Da questo deriva che, supponendo che $ r $ abbia $ n $ figli, vi saranno al pi\`u $  n $ alberi radicati in ciascuno di essi tali che: $ |T_i| \ge |T_{i+1}|$ \ $ \forall i = 1,\dots, n-1 $.\mbox{}\\\\

	\begin{figure}[htbp]
	\centering
	\includegraphics[width=5cm]{capitolo3/grafo3}
	\caption{Esempio di albero $ T $ radicato in $ r $, con $ k $ nodi, valido come input per l'algoritmo \ref{algoritmo1}} 
	\label{fig:3}
\end{figure}
\mbox{}\\

\begin{algorithm}[H]
	\label{algoritmo1}
	\SetAlgoLined
	\caption{Insiemi $ k $-bilanciati}
	\textbf{input} : Albero $ T $ con $ k $ nodi, fattore di bilanciamento $ f(k-1) $;\\
	$ T' , T''\longleftarrow $ due alberi inizialmente senza nodi;\\ 	
	\For{$ i = 1,\dots,n $}
	{
		\If{$ \sum_{j=1}^{i}{|T_j|} > f(k-1) $}
		{
			$ T'\longleftarrow $ sottoalbero di  di $ T $ indotto da $ \{r\} \cup \bigcup_{j=1}^{i-1}V(T_j)$.\\
			$ T''\longleftarrow $ sottoalbero di $ T $ indotto da $ V(T) \diagdown T'$.
		}
		 
}
	\textbf{return} ($ T',T'' $).

\end{algorithm}\mbox{}\\

Una volta concluso l'algoritmo \ref{algoritmo1} si avr\`a una coppia $ (T',T'') $ tale che : $ \max (|T'|,|T''|) \le f(k-1)=f(k) $. 
Viene pertanto rispettata la definizione \ref{lemmaDeco}, perci\`o si ottiene una decomposizione $ f(k) $-bilanciata.

In base a tutte le nozioni fino ad ora discusse si pu\`o dare il seguente risultato\mbox{}\\
.


\newtheorem{teorema1}[definizione]{Teorema}
\begin{teorema1}
	\label{teorema1 cap3 sez1}
Per ogni albero T di k nodi esiste un nodo r di T  tale che l'albero $T_r$, ottenuto radicando T in r, ammette una decomposizione $ (\lfloor \frac{2}{3}(k-1) \rfloor + 1)$-bilanciata ed, inoltre, sia $ (|T'|,|T''|)$ tale decomposizione risulta che il $ \max \{|T'|,|T''|\} \ge 2+ \left\lfloor \frac{(k-1)}{3}\right\rfloor $
\end{teorema1}\mbox{}
\begin{proof}
	
	Sia $ T $ un albero di $ k $ nodi,con $ k>2 $ (per $n\le2$ la propriet\`a \`e banalmente vera) \\
	La prima operazione da compiere \`e individuare il nodo $ r $ di $ T $ su cui si andr\`a poi a radicare il nuovo albero $ T_r $.\\
	Si suppone che tale nodo sia un centroide dell'albero $ T $ quindi si applica l'algoritmo di Jordan per la ricerca del centroide. Senza perdita di generalit\`a si suppone di avere un unico nodo centroide, indicato con $ r $.\\
	Se il centroide $ r $ trovato non corrisponde alla radice dell'albero $ T $, si modifica $ T $ radicandolo in $ r $ e l'albero cos\`i ottenuto verr\`a indicato con $ T_r $. \\ 
	Inoltre, si suppone che i sottoalberi radicati nei figli di r siano ordinati in maniera non crescente rispetto alla loro dimensione.\\
	Si applichi a $ T_r $ l'algoritmo \ref{algoritmo1} precedentemente descritto per definire se \`e possibile ottenere una decomposizione $ f(k) $-bilanciata, con $ f(k) =
	(\lfloor \frac{2}{3}(k-1) \rfloor + 1  $.\\
	Sia $ \{T_i \ | \  i=1,\dots,n\} $ l'insieme dei sottoalberi radicati negli $ n $ figli di $ r $ e si consideri il primo valore di $ i $ tale che l'istruzione $ if $ di riga $ 4 $ nell'algoritmo \ref{algoritmo1} risulti vera (si noti che tale valore di $ i $ esiste sempre dal fatto che per $ i = n $ la condizione \`e verificata).\mbox{}\\\\
	Sia 
	\[ S = \sum_{i=1}^{n}{|T_i|} = (k-1 ) \]\\
	e sia
	\[ x = \sum_{j=1}^{i-1}{|T_j|} \]\\
	Distinguiamo due casi
	\begin{itemize}
	\item $\textbf{ i>2 }$ Si ha che
	\\ 
	\begin{equation}\label{1}
		x+|T_i| > \frac{2}{3}\cdot S
	\end{equation}
\\
	Inoltre sapendo che per costruzione
	\\
	
	\begin{equation}\label{2}
	|T_i| \le \frac{S}{i} \le \frac{S}{3}	
	\end{equation}
\\
		
	Sottraendo la disequazione \eqref{2} alla \eqref{1} si ottiene che 
	\\
	\begin{equation}\label{3}
	x > \frac{2}{3}\cdot S - \frac{S}{3} = \frac{S}{3}	
	\end{equation}
\\
 	\item $ \textbf{i=2} $ Anche in questo caso come nel precedente vale la disequazione \eqref{1}.\\
 	Inoltre, essendo $ i = 2 $ per costruzione dell'albero $ T $ si pu\`o dire che
 	
 	\begin{equation}\label{4}
 	x = |T_1| \ge |T_2| = |T_i|
 	\end{equation}
 	
 	Pertanto, sfruttando la disequazione \eqref{4} combinata con la \eqref{1} si ha che
 	\begin{equation}\label{5}
 	2x > \frac{2}{3} \cdot S \Rightarrow x > \frac{S}{3}
 	\end{equation}
	\end{itemize}\mbox{}\\


Per entrambi i casi otteniamo che 
\\
\[ x > \frac{S}{3} \Rightarrow x \ge \left\lfloor \frac{S}{3}\right\rfloor  + 1 \]
\\

Pertanto si avr\`a che 
\\
\[ \left\lfloor \frac{S}{3}\right\rfloor  + 1 \le x \le \left\lfloor \frac{2}{3}\cdot S \right\rfloor \] 
\\

Quindi 
\\
\begin{equation}\label{5}
|T'| = 1+x \le 1 + \left\lfloor \frac{2}{3}\cdot S \right\rfloor = 1 + \left\lfloor \frac{2}{3} \cdot (k-1) \right\rfloor	
\end{equation}
\\
\begin{equation}\label{6}
|T''| = 1 + S - x = 1+S-1 - \left\lfloor \frac{S}{3}\right\rfloor = \left\lceil \frac{2}{3}\cdot S \right\rceil = \left\lceil \frac{2}{3} \cdot (k-1) \right\rceil 	
\end{equation}
\\
Inoltre
\\
\[ |T'| = 1+ x \ge 1+ (1 +  \left\lfloor \frac{S}{3}\right\rfloor ) = 2 +  \left\lfloor \frac{(k-1)}{3}\right\rfloor\]
\\
Poich\`e
\\
 	 \[1 + \left\lfloor \frac{2}{3} \cdot (k-1) \right\rfloor \ge \left\lceil \frac{2}{3} \cdot (k-1) \right\rceil \]
 	 \\

Possiamo concludere che
\\

\[2 +  \left\lfloor \frac{(k-1)}{3}\right\rfloor \le \max\{|T'|,|T''|\} \le 1 + \left\lfloor \frac{2}{3} \cdot (k-1) \right\rfloor \]
 \\
ottenendo perci\'o una decomposizione $ (\lfloor \frac{2}{3}(k-1) \rfloor +1)$-bilanciata, tale che $ \max \{|T'|,|T''|\} \ge 2+ \left\lfloor \frac{(k-1)}{3}\right\rfloor $	 
 	 
\end{proof}\mbox{}\\
 
 	Nel caso in cui si abbiano due centroidi, la scelta su quale radicare l'albero  \`e deterministica.\\
 	Nel caso in cui gli alberi ottenuti radicando $ T $ in ognuno di essi abbiano strutture differenti, viene scelto quello che tra i due ha una struttura pi\`u piccola.
 	Altrimenti se gli alberi ottenuti sono identici, viene scelto uno dei due in maniera arbitraria.
 	
\section{Algoritmo}
\label{cap:3 par:2}
In questa sezione si vede come \`e stato utilizzato il risultato del paragrafo \ref{cap:3 par:1} per ottimizzare e migliorare l'algoritmo \ref{algoritmo} descritto nel capitolo \ref{cap 2} paragrafo \ref{section1}.\\
Molto brevemente, quello che si faceva precedentemente era la seguente cosa.
Dato $ T_C = (T,C) $, con $ T $ un albero colorato radicato di $ k $ nodi i cui colori giacciono in $ C $, si procedeva a calcolare il numero di occorrenze $ c(T_C,v) $ \ $ \forall v \in V $ dividendo idealmente $ T $ in due sottoalberi, $ T' $ e $ T'' $, con $ T'' $ il sottoalbero radicato in $ u $ uno dei figli della radice $ v $ di $ T $ e $ T'$ il sottoalbero di $ T $ radicato in $ v $ contenente $ |T | - |T'' | $ nodi. 
Si procedeva al conteggio delle occorrenze di $ T_C $ nel seguente modo
\[	c(T_c,v)=\frac{1}{\beta_T}\sum_{(u,v)\in E}\sum_{\substack{C' \subset C \\C'' = C \setminus C' \\ |C'|=|T'|, |C''| = |T''|}}c(T'_{C'},v)\cdot c(T''_{C''},u)\]   

In questa nuova versione, invece, la divisione dell'albero $ T_C $ non \`e pi\`u scelta in maniera ideale, ma bens\`i vengono presi due alberi $ T'_{C'} $ e $ T''_{C''} $ tali che la decomposizione di $ T_C $ risulti bilanciata.
Inoltre, a differenza della precedente versione, gli alberi $ T'_{C'} $ e $ T''_{C''} $ saranno radicati entrambi nello stesso nodo $ v $ su cui sar\`a poi radicato $ T_C $.



