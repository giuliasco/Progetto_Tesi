\chapter{Decomposizioni bilanciate}

Si vuole andare a dimostrare in questo paragrafo, che dato un albero T \`e sempre possibile ricavare una scomposizione bilanciata dell'albero.
\\
Prima di poter enunciare il teorema e dimostrarlo occorre dare delle nozioni preliminari.
Innanzitutto va definito cosa si intende per scomposizione bilanciata di un albero.
\newtheorem{definizione}{Definizione}[chapter]
\newtheorem{lemma}[definizione]{Lemma}
\begin{definizione}
Sia $T_r$ un albero radicato nel nodo r , con k nodi.
Diremo che la coppia (A,B), dove  A e B sono insiemi contenenti i nodi di $T_r$, \`e una decomposizione per l'albero se:
\begin{itemize}
	\item $| A |\cup| B | = k$
	\item $A \cap B = r$.
\end{itemize}
\end{definizione}

\begin{lemma}
Affinch\`e una scomposizione risulti bilanciata dovr\`a risultare che:
\begin{equation*}
	\max{ \{|A| , |B| \} }  \le  f(k)
\end{equation*}


dove f(k) rappresenta il fattore di bilanciamento, varia in funzione al numero dei nodi dell'albero ed \`e pari a:
\begin{equation*}
f(k) = \left\lceil \frac{2}{3}  k  \right\rceil
\end{equation*}
\end{lemma}

\paragraph{Dimostrazione}\mbox{}\\\\
\begin{proof}
	Dimostrazione.....
\end{proof}

\begin{definizione}
Per ogni nodo v di un albero T, le diramazioni di T  rispetto a v, sono tutti i sottoalberi massimali, radicati nei figli di v. 
Sia $\alpha(v)$ il numero di nodi della massima diramazione di v. 

Un nodo u di un albero T con n nodi, \`e un nodo centroide se $\alpha(u)\le\frac{n}{2}$.

Il centroide di un albero non \`e necessariamente unico, infatti Jordan \cite{jordan1869assemblages}  ha dimostrato che o (i) T ha un singolo centroide v e $\alpha(v) < \frac{n}{2}$ oppure (ii) t ha due nodi centroidi (adiacenti) $v_1$ e $v_2$ tali che $\alpha(v_1) = \alpha(v_2) = \frac{n}{2}$, in questo caso il numero di nodi n \`e pari.
\end{definizione}

Esistono diversi algoritmi per la ricerca del centroide, quello da noi usato \`e l'algoritmo di Jordan che  ha una complessit\`a temporale lineare al numero di nodi O(n). \\
Il primo passo da fare \`e determinare $\alpha(v)$ $\forall v \in T$.\\
Per poter calcolare $\alpha$ $\forall v \in T$, inizialmente si effettua una visita DFS (Depth First Search) dell'albero a partire dalla radice, in modo da definire la cardinalit\`a di ogni possibile sottoalbero a partire dalla radice, procedendo su ogni suo figlio .
I sottoalberi ottenuti dai nodi foglia e dal nodo radice, sono alberi banali che avranno rispettivamente cardinalit\`a 1 e $|T|$. \\
A questo punto si pu\`o procedere con l'individuazione di $\alpha (v)$, ossia $\forall v$ della diramazione con un maggior numero di nodi.

Per farlo, si considera ogni nodo v di T e si calcola la cardinalit\`a dei sottoalberi radicati in ognuna delle sue possibili diramazioni. 
$\alpha (v)$ sar\`a il valore massimo tra tutte le quantit\`a individuate.
\\
\newtheorem{esempio}[definizione]{Esempio}
\begin{esempio}\mbox{}\\
Si prenda l’albero T in figura \ref{fig:1}.\\ 
T ha otto nodi e per ogni nodo \`e indicata la cardinalit\`a del sottoalbero radicato in esso.
\\
	\begin{figure}[htbp]
	\centering
	\includegraphics[width=10cm]{capitolo3/1}
	\caption{}
	\label{fig:1}
\end{figure}

Perci\`o per il nodo {\color{red} 1}, si avr\`a che:
\begin{equation*}
\alpha(1) = max {\{ |T_4| , |T_5| , (|T| - |T_1| )\}} = \max{\{1, 1, 5\} }  = 5
\end{equation*}

Dove $T_4$ rappresenta il sottoalbero di T radicato nel nodo 4, analogamente, mentre con l’ultimo valore rappresentiamo la cardinalit\`a del sottoalbero radicato in 0 e contenente tutti i nodi restanti non inclusi nel sottoalbero di T radicato in 1. 
\end{esempio}
Una volta determinato $\alpha(v)$ $\forall v \in T$, si procede alla ricerca del centroide, che sar\`a il nodo di T per cui vale la seguente disuguaglianza:
\begin{equation*}
\alpha(v) \le \left\lfloor\frac{n}{2} \right\rfloor
\end{equation*}

\paragraph{Esempio 2}\mbox{}\\
Si consideri l’albero T in figura \ref{fig:2} per la ricerca del nodo centroide.
Per prima cosa su ogni nodo di T, numerati da 0 a 10, viene calcolato $\alpha(v)$. \\\\
Quello che si otterr\`a sar\`a:

%$\alpha(0) = 6$; $\alpha(1) = 5$; $\alpha(2) = 7$; $\alpha(3) = 10$; $\alpha(4) = 7$; $\alpha(5) =  10$ $\alpha(6) = 9$; $\alpha(7) = 10$; $\alpha(8) = 10$; $\alpha(9) = 10$; $\alpha(10) = 10$.


\begin{center}
	\begin{tabular}{ c c c c c  }
		$\alpha(0) = 6$ & & $\alpha(1) = 5$ & & $\alpha(2) = 7$ \\ 
		$\alpha(3) = 10$ && $\alpha(4) = 7$ &&  $\alpha(5) =  10$ \\  
		$\alpha(6) = 9$ && $\alpha(7) = 10$ && $\alpha(8) = 10$ \\
		 $\alpha(9) = 10$ && $\alpha(10) = 10$ &&
	\end{tabular}
\end{center}
Poich\'e $ \left\lfloor\frac{n}{2} \right\rfloor = \left\lfloor \frac{11}{2} \right\rfloor = 5$, basta verificare per quale nodo v di T , vale che $\delta(v) \le \left\lfloor\frac{n}{2} \right\rfloor$.
L’unico nodo per cui tale disuguaglianza risulta vera \`e il nodo 1, infatti $5\le 5$, e sar\`a l’unico centroide dell’albero T (figura \ref{fig:3}).
	\begin{figure}[!htb]
	\begin{minipage}{0.48\textwidth}
	\centering
	\includegraphics[width=5.26cm]{capitolo3/grafo1c}
	\caption{Rappresentazione dell'albero T}
	\label{fig:2}
	\end{minipage}\hfill
	\begin{minipage}{0.48\textwidth}
			\centering
		\includegraphics[width=5.8cm]{capitolo3/grafo2}
		\caption{Rappresentazione del centroide in T}
		\label{fig:3}
	\end{minipage}
\end{figure}\\
Come gia accennato l’algoritmo ha complessit\`a lineare sul numero dei nodi.
Infatti, occorre calcolare il grado di ogni nodo e successivamente verificare che ci siano le condizioni affich\`e risulti un centroide e si ha:

\begin{equation*}
\sum_{v}^{}(1 + \delta(v)) = n + \sum_{v}^{} \delta(v) = n+n-1= 2n-1 =O(n)
\end{equation*}\\
In base a tutte le nozioni illustrate in precedenza possiamo passare a ad enunciare il  teorema di seguito.

\paragraph{Teorema}\mbox{}\\
Per ogni albero T di k nodi esiste un nodo r di T , tale che l’albero $T_r$, ottenuto radicando T in r ammette una decomposizione bilanciata.

\paragraph{Dimostrazione}\mbox{}\\
Sia un albero T con pi\`u di due nodi(per $n\le2$ caso banale). \\
La prima operazione da compiere \`e l’individuazione del nodo r di T, su cui si andr\`a poi a radicare l’albero.
Banalmente, per come \`e stato definito, il nodo che si cerca, non \`e altro che  un centroide dell’albero T, quindi si applica l’algoritmo precedentemente descritto per la sua ricerca.
Una volta trovato, questo sar\`a il nuovo nodo su cui sar\`a radicato l’albero T, che da questo punto sar\`a indicato con $T_r$.
\\ 
Inoltre supponiamo, senza perdere di generalit\`a, che i sottoalberi radicati nei figli di r siano ordinati in maniera non crescente rispetto alla loro dimensione.\\
\`E possibile ottenere una scomposizione, (A,B), dei k nodi di $T_r$,  tale che un insieme, ad esempio A, contenga al massimo i $\left\lceil \frac{2}{3} \right\rceil$ dei nodi dell’albero e B il restante di essi.\\
Ovviamente questo algoritmo terminer\`a, poich\'e il numero di nodi \`e finito.
Inoltre l’insieme con il maggior numero di elementi non conterr\`a pi\`u dei $\left\lceil \frac{2}{3} \right\rceil$ del totale. \\
Per dimostrarlo si osserver\`a che A deve contenere almeno $\frac{1}{3}$ dei nodi totali. \\
Si suppone di aver inserito nell’insieme A una certa quantit\`a di elementi, sia un numero pari a $\frac{2}{3}$k. \\
Sia x il primo elemento non in A e sia i la sua posizione (figura \ref{fig:4}).
	\begin{figure}[htbp]
	\centering
	\includegraphics[width=6cm]{capitolo3/3}
	\caption{}
	\label{fig:4}
\end{figure}

Si possono verificare  due casi:
\begin{itemize}
	\item (i=2)  L’insieme A \`e formato da un unico elemento y:
		\begin{figure*}[htbp]
		\centering
		\includegraphics[width=6cm]{capitolo3/4}
			\caption{}
	\end{figure*}\\
Per come \`e costruzione di A, si avr\`a certamente che:
\\
\begin{equation}
 y + x > \frac{2}{3}k
\end{equation}
\\
Dividendo entrambi i membri di (1) per  due, si ottiene:
\\
\begin{equation}
\frac{ x + y }{2} > \frac{k}{3}  
\end{equation}
Si nota che $\frac{ x + y }{2} $ rappresenta esattamente il valore medio. \\
Dall’ordinamento dei sottoalberi di $T_r$, risulta che $y \ge x$ perci\`o si avr\`a che:
\begin{equation}
y \ge \frac{ x + y }{2} 
\end{equation}
unendo la (2) e la (3)  si ottiene:
\begin{equation*}
y > \frac{k}{3}  
\end{equation*}
\item ($i\ge3$) In A vi sono almeno due elementi. Sia s il valore ottenuto dalla loro somma 
	\begin{figure*}[htbp]
	\centering
	\includegraphics[width=6cm]{capitolo3/5}
		\caption{}
\end{figure*}\\
Si avr\`a che:
\begin{equation}
s+ x > \frac{2}{3}
\end{equation}
Inoltre, per costruzione:
\begin{equation}
x \le \frac{k}{3}
\end{equation}
Sottraendo la (5) alla (4), ammissibile poich\'e rispetta le regole delle disequazioni, si otterr\`a:\\
\begin{equation}
	s + x - x > \frac{2}{3} k - \frac{k}{3} \hspace{0.3cm} \text{    ossia    }\hspace{0.3cm} s > \frac{k}{3}
\end{equation}
Perci\`o gli insiemi ottenuti dalla scomposizione di $T_r$ avranno cardinalit\`a compresa tra $\frac{1}{3} k$ e $\frac{2}{3} k$, garantendo cos\`i delle decomposizioni bilanciate.\\ 
Nel caso in cui si abbiano due centroidi, la scelta su quale radicare l’albero  \`e deterministica e  viene fatta prendendo quello che, tra i due, ha un $k(v)$ minore rispetto alla relazione d’ordine precedentemente fornita.


\end{itemize}