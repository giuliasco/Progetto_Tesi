\chapter{Conclusioni}
Nel presente lavoro di testi \`e stato descrittao  un algoritmo per il conteggio dei $ k $-treelet di un grafo basato sulla tecnica del Color Coding (capitolo \ref{cap 2}).
È stato mostrato come tale algoritmo pu\`o essere sfruttato per la ricerca e il conteggio approssimato di tutti i $ k $-treelet colorati presenti in un grafo $ G=(V,E) $.
Si \`e visto come  l'algoritmo sia stato sviluppato con un approccio basato sulle unioni tra alberi piuttosto che sulle scomposizioni e sono state discusse le scelte implementative sviluppate per rappresentare i treelet colorati e i conteggi associati.

Sucessivamente si \`e provveduto a introdurre un'ottimizzazione a tale algoritmo.\\
Per fare ci\`o \`e stato necessario introdurre alcune nozioni fondamentali come ad esempio la nozione di centroide e di decomposizioni bilanciate, l'algoritmo risultante \'e stato descritto nel capitolo \ref{cap:3}.
Anche per questo nuovo algoritmo, come per il precedente, il conteggio dei $ k $-treelet basandoci sull'unione piuttosto che sulle decomposizioni.

Infine sono stati messi a confronto i risultati ottenuti con l'esecuzione di entrambe le implementazioni, discutendo l'accuratezza dei risultati ed il risparmio in termini di tempo che si \`e ottenuto usando l'algoritmo ottimizzato.

Come detto inizialmente, il lavoro di tesi si \`e concentrato solo sul conteggio $ k $-treelet (non indotti) di un grafo $ G $. Cionondimeno, in \cite{bressan2018motif,bressan2019motivo} \`e mostrato come sia possibile campionare (e contare) le occorrenze dei $ k $-graphlet di interesse nel grafo in input $ G $ avendo a disposizione un algoritmo in grado di campionare i $ k $-treelet di $ G $.
A tale scopo le tecniche descritte in \cite{bressan2018motif,bressan2019motivo} possono essere facilmente adattate per restituire un $ k $-treelet di $ G $ ben colorato uniformemente a caso a partire dai conteggi $ c(T_C,v) $ calcolati dall'Algoritmo \ref{algoritmo2} permettendo così di applicare le ottimizzazioni descritte nella tesi anche al problema, più generale, della stima delle occorrenze dei motif in un grafo.

