\chapter{Conclusioni}
Si pu\`o concludere questa tesi riassumendo brevemente quanto visto fino ad ora.\\
Innanzitutto \`e stato definita la fase di costruzione dell'algoritmo del Color Coding (capitolo \ref{cap 2}).\\
\`E stato mostrato come tale algoritmo pu\`o essere sfruttato per la ricerca e il conteggio di $ k $-treelet colorati presenti in un grafo $ G=(V,E) $.
Si \`e visto come, in questa tesi, sia stato sviluppato con un approccio bottom-up piuttosto che top-down.\\
Sono state discusse le scelte implementative sviluppate per rappresentare i treelet colorati e i conteggi associati.

Sucessivamente si \`e provveduto a introdurre un'ottimizzazione a quanto detto prima.\\
Per fare ci\`o \`e stato necessario introdurre alcune nozioni fondamentali come ad esempio la nozione di centroide e di decomposizioni bilanciate.
Una volta fatto ci\`o si \`e modificato l'algoritmo del color coding, sulla base delle nozioni date, definendo cos\`i un nuovo algoritmo nel capitolo \ref{cap:3}.\\
Anche per questo nuovo algoritmo si \`e discusso sulla scelta di valutare i $ k $-treelet con un approccio bottom-up, piuttosto che top-down e si sono analizzate le scelte implementative aggiunte rispetto alla precedente versione.\\
Alla fine sono stati messi a confronto i risultati ottenuti con l'esecuzione di entrambe le implementazioni, discutendo il guadagno, in termini di tempo, che si \`e avuto usando il secondo algoritmo.

\section{Sviluppi futuri}
Come detto inizialmente, in questa tesi \`e stata sviluppata e ottimizzata solo una piccola parte di quello che \`e realmente l'algoritmo del color coding.\\
Infatti si \`e limitata la ricerca ai $ k $-treelet (non indotti) di un grafo $ G $.

Estensione naturale a tale ricerca \`e quella che concerne la ricerca dei graphlet, indotti in $ G $.\\
Pertanto un possibile sviluppo sar\`a quello di aggiungere all'algoritmo ottimizzato una fase di campionamento per la ricerca dei graphlet indotti in $ G $, allo scopo di garantire cos\`i dei tempi di esecuzione pi\`u rapidi. 
