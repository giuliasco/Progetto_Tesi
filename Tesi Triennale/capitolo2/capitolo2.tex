\chapter{Color Coding}

In questo capitolo verr\`a descritta la tecnica del Color Coding introdotta da Alon, Yuster e Zwick  in \cite{alon1995color} , la sua estensione algoritmica e l'utilizzo che ne \`e stato fatto in questa tesi , per il calcolo della frequenza di k-treelet in un grafo.

\section{Tecnica del Color Coding}

La tecnica del Color Coding fu presentata per la prima volta nel 1995, da Alon, Yuster e Zwick \cite{alon1995color}.
In generale, dato un grafo $G = (V,E)$, il problema dell'isomorfismo dei sottografi  di $G$ \`e un problema $NP-completo$.
Il metodo del Color Coding permette di risolvere sottocasi di questo problema in tempo polinomiale.

La tecnica del $color$-$coding$ \`e una tecnica randomizzata. Dati un grafo $ G = (V,E) $ ed uno $H = (V_H, E_H)$, i vertici $V$ di $G$, in cui verr\`a cercato un  sottografo isomorfo ad $H$, sono colorati casualmente di $k = |V_H|$ colori. Se $|V_H| = O(\log(V))$, allora, tutti i vertici del sottografo di $G$ isomorfo ad $H$, se esiste, saranno colorati da colori distinti.   

Probabilmente i sottocasi pi\`u interessanti e semplici di problemi di sottografi isomorfi sono i seguenti.
Dato un grafo $G=(V,E)$ (diretto o non) e un numero $k$, $G$ pu\`o contenere un percorso(diretto o non) di lunghezza $k$? $G$ pu\`o contenere un ciclo (diretto o non) di lunghezza $esattamente$ $k$?

Sia $G=(V,E)$ un grafo diretto o indiretto. Si consideri il problema di trovare un percorso semplice (diretto o no) di lunghezza $k-1$ in $G$. Data una colorazione dei vertici di $G$ con $k$ colori. Un percorso in $G$ \`e detto $colorato$ se ogni suo vertice \`e colorato da un colore distinto. Un percorso colorato \`e chiaramente semplice. Ogni percorso semplice di lunghezza $k-1$, d'altro canto, ha una possibilit\`a di diventare colorato di $k!/k^k>e^{-k}$. Si noti che questa quantit\`a \`e solo esponenzialmente piccola in $k$. Quello che si andr\`a per\`o a vedere, \`e il tempo  necessario per trovare un percorso colorato di lunghezza $k-1$ in $G$, se esiste, o tutte le coppie di vertici connesse da percorsi colorati di lunghezza $k-1$ in $G$.

\newtheorem{lemma}{Lemma}[section]
\begin{lemma}
	\label{lemma1}
	Sia $G=(V,E)$ un grafo diretto o indiretto e sia $c:V\rightarrow\{1,......,k\}$ una colorazione dei suoi vertici con $k$ colori. Un percorso colorato in $G$, se esiste, pu\`o essere trovato, nel caso peggiore, in $2^{O(k)} \cdot E$
\end{lemma}
\begin{proof}[\textbf{Dimostrazione}]

	Per prima cosa viene descritto un algoritmo di complessit\`a temporale $2^{O(k)} \cdot E$, che preso in input il grafo $G=(V,E)$, la colorazione $c:V\rightarrow\{1,......,k\}$ ed un vertice $s\in{V}$ restituisce, se esiste, un percorso colorato di lunghezza $k-1$ che parte da $s$.
	Per trovare un percorso colorato di lunghezza $k-1$ in $G$ che parte in un qualche punto si aggiunge semplicemente a $G$  un nuovo vertice $s'$, colorato con un nuovo colore $0$ e connesso con archi a tutti i vertici di $V$. A questo punto si cerca un percorso colorato di lunghezza $k$ che parte da $s'$.
	Un percorso colorato di lunghezza $k-1$ che inizia ad un specifico vertice $s$ \`e trovato usando un approccio di programmazione dinamica.
	Si suppone di aver gi\`a trovato per ogni vertice $v\in{V}$ i possibili insiemi di colori dei percorsi di lunghezza $i$ che connettono $s$ a $v$.
	Si noti che non vengono registrati tutti i percorsi colorati da $s$ a $v$, ma solo l'insieme dei colori di ogni percorso.
	Per ogni vertice $v$ si ha perci\'o una collezione di al pi\`u $\binom {k}{i}$ insiemi di colori.
	Si ispeziona ogni sotto insieme $C$ che appartiene alla collezione di $v$, e ogni arco $(u,v)\in{E}$. 
	Se $c(u)\notin{C}$, si aggiunge l'insieme $C\cup\{c(u)\}$ alla collezione di $u$ degli insiemi di colori dei percorsi colorati di lunghezza $i+1$.
	Il grafo $G$ contiene un percorso colorato di lunghezza $k-1$ rispetto alla colorazione $c$ se e solo se la collezione finale, che corrisponde ai percorsi di dimensione $k-1$ di almeno un vertice \`e non vuota.
	Il numero di operazioni svolte dall'algoritmo appena descritto sono al pi\`u $O(\sum_{i=o}^{k} i \binom{k}{i} \cdot |E|)$ che \`e chiaramente $O(k2^{k}\cdot{E})$.
\end{proof}

\begin{lemma}
	\label{lemma2}
	Sia $G=(V,E)$ un grafo diretto o indiretto e sia $c:V\rightarrow\{1,......,k\}$ una colorazione dei suoi vertici con $k$ colori. Tutte le coppie di vertici connessi da percorsi colorati di lunghezza $k-1$ in $G$ pu\`o essere trovato in tempo $2^{O(k)} \cdot{VE}$ o $2^{O(k)}\cdot {V^{\omega}}$ nei casi peggiore.
\end{lemma}

\begin{proof}[\textbf{Dimostrazione}]
	L'algoritmo con complessit\`a $2^{O(k)} \cdot{VE}$  \ 
	\`e ottenuto dalla semplice esecuzione dell'algoritmo descritto nella prova del Lemma \ref{lemma1}, un numero $|V|$ di volte, una volta per ogni vertice iniziale.
\end{proof}
	
