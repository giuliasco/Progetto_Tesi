\chapter{Introduzione}

La terapia renale di sostituzione (Renal Replacement Therapy, \textbf{RRT}) viene utilizzata sui pazienti critici in contesti di terapia intensiva che sviluppano insufficienza renale acuta (Acute Kidney Injury, \textbf{AKI}).
L'AKI è una sindrome clinica caratterizzata da un improvvisa riduzione della funzione renale risultante da un accumulo di fluidi, creatinina e urea ed altri prodotti di scarto. L'incidenza dell'AKI varia a seconda della popolazione studiata e della definizione usata. Nel 2012 le linee guida \textit{"Kidney Disease: Improving Global Outcomes"} (KDIGO) hanno definito l'AKI come un aumento del livello di creatinina sierica di 0,3 mg/dL (26,5 mmol/L) o più entro 48 ore, un livello di creatinina sierica che  aumenta di almeno 1,5 volte il valore basale nei 7 giorni precedenti o un volume di urina inferiore a 0,5 ml /kg di peso corporeo all'ora per 6 ore.
Circa il 5-7\% dei pazienti ospedalizzati sviluppa l'AKI durante la degenza ospedaliera, questa incidenza aumenta al 25\% tra i pazienti critici nell'unità di terapia intensiva\cite{tolwani2012continuous,uchino2005acute}.
È stato riportato un tasso di mortalità superiore al 50\% per i pazienti con l'AKI e insufficienza multiorgano\cite{uchino2005acute}.
In assenza di terapie farmacologiche efficaci, l'AKI viene solitamente gestita attraverso trattamenti di supporto incentrati sull'ottimizzazione dell'equilibrio dei  fluidi, sulla prevenzione o sul trattamento dei disturbi elettrolitici e acido-base, sull'aggiustamento del dosaggio dei farmaci renali ed evitando un danno renale emodinamico e nefrotossico secondario.
Oltre a queste terapie conservative, la RRT è essenzialmente l'unico metodo efficace per la gestione dei pazienti critici con l'AKI grave\cite{villa2015renal}. 
L'RRT include la dialisi (emodialisi e dialisi peritoneale), l'emofiltrazioni e la emodiafiltrazione che sono differenti metodi di filtraggio del sangue con o senza macchinari\cite{tiglis2022overview}.

Ci possono essere tre tipi di RRT:
\begin{enumerate}
	\item \textbf{Continuos renal replacement therapy (CRRT)}  è la forma  più usata in terapia intensiva. Il beneficio della CRRT per pazienti critici è che agisce lentamente (generalmente oltre 24 ore per diversi giorni) permettendo la rimozione dei fluidi e delle tossine uremiche in eccesso con meno rischi di complicazioni ipotensive\cite{karkar2019continuous}.
	\item \textbf{Intermittent renal replacement therapy (IRRT)}  eseguita per meno di 24 ore in un periodo qualunque, da due a sette volte a settimana\cite{rabindranath2007intermittent}.
	\item \textbf{Hybrid Renal Replacement Therapy  (HRRT)}  è la combinazione di entrambe le tecniche descritte in precedenza.
\end{enumerate}
Un'altro importante fattore da tenere in considerazione sono i differenti emofiltri (cioé le differenti membrane) utilizzati nelle terapie. \\
A seguire vengono presentati alcuni dei più diffusi:
\begin{itemize}
	\item \textit{\textbf{AN69/AN69ST}} è un emofiltro a membrana in poliacrilonitrile trattata in superficie, consente il fissaggio irreversibile dell'eparina alla membrana filtrante\cite{doi2017associations}. 
	
	\item \textit{\textbf{oXiris}} è una membrana dializzante modificata ricoperta con eparina, capace di rimuovere frammenti di endotossine\cite{turani2019continuous}.
	
	\item \textit{\textbf{SepteX}} caratterizzato da pori di dimensioni maggiori rispetto ai filtri convenzionali \cite{honore2013newly}.
	
\end{itemize}


La medicina evidence based, si basa su studi clinici che valutano l'efficacia dei trattamenti che poi verranno utilizzati nella pratica clinica. Spesso però risulta molto difficile validare l'efficacia dei risultati ottenuti, soprattutto se i trattamenti considerati si applicano a pazienti particolarmente critici come quelli di rianimazione, infatti è poco credibile supporre che un singolo trattamento, come ad esempio quello di purificazione ematica extracorporea possa modificare sensibilmente l'outcome di questi pazienti, che sono per loro natura estremamente complessi. 
Questo evidenzia anche come questi tipi di trattamenti non abbiano un livello di evidenza sufficiente per essere raccomandati nella pratica clinica da linee guida.

Pertanto quello che spesso avviene è che i medici utilizzino i trattamenti in maniera casuale, pensando che ci sia una ragione fisiopatologica per utilizzarli piuttosto che una direttiva che li raccomandi.

Ad esempio, si consideri il caso della purificazione ematica extracorporea mediante emoperfusione con polimixina B, per l'emoassorbimento di endotossine.
Negli ultimi quarant'anni sono stati svolti molteplici trial su questo trattamento e tutti fallimentari. In un primo trial si sono considerati tutti quei pazienti con shock settico, dopodiché si è ristretta la popolazione poiché si è notato che poteva esserci un ruolo nelle sepsi addominali, sostenute da gram negativi ed effettivamente questo poteva aver senso perché questo tipo di trattamenti va a rimuovere una specifica tossina prodotta dai batteri gram negativi.

Anche questo studio però si è rivelato fallimentare, perché solo la sottopopolazione con valori di endotossinemia elevati beneficiava effettivamente del trattamento. Questo ha senso poiché essendo proprio trattamenti per l'eliminazione delle endotossine, chiaramenti i pazienti con valori più alti andavano incontro ad una purificazione più efficace con un miglioramento clinico sensibile. 
Quindi a questo punto arruolarono i pazienti con sepsi addominali e con valori di endotossina ematica circolante per poi scoprire, anche in questo caso che la sottopopolazione di pazienti che beneficiava più di altri era quella con un determinato valore di SOFA score. 
Ossia quelli che non erano così gravi da avere un valore di SOFA molto alto e che sarebbero morti nonostante il trattamento ma neanche quelli poco gravi con un basso valore e che sarebbero sopravvissuti comunque.

Questo esempio permette di dire come l'evidence base medicine nella medicina moderna, in pazienti così complessi, con trattamenti così complessi fondamentalmente va a tentivi. 

Ed è a questo punto che entrano in gioco i registri. 
Poiché questi tipi di trial randomizzati, sono estremamente costosi e quasi sempre fallimentari, la cosa più semplice da fare a questo punto, è andare a capire e guardare cosa fa la popolazione, cosa fa il medico nella sua normale pratica clinica e, con un processo a posteriori, determinare se esistono delle sottopopolazioni di pazienti che sembrano beneficiare più di altri di questi trattamenti, così da invertire il processo di formazione delle prove. 
Quindi non identificare una popolazione, prescrivere un trattamento standard e valutare l'outcome, ma è esattamente rigirarlo cioé andare a capire quali sono i pazienti che hanno avuto esito positivo, capire come siano stati trattati. Questa identificazione dei pazienti può naturalmente essere paragonata ad una cluster analisi \cite{villa2019oxirisnet}. 

Per tutti questi motivi nel 2019, un gruppo di medici dell'Università di Firenze, ha avviato il  \textbf{Registro ARRT} (Acute Renal Replacement Therapy), con lo scopo di analizzare e studiare tutti quei casi di RRT acute.

Nel trial sono coivolti numerosi centri in tutta Italia, con circa 700 pazienti totali.

\section{Contributo della tesi}

In questo lavoro di tesi, sono stati estratti i dati dal registro ARRT tenendo traccia solo di quei pazienti sottoposti al trattamento CRRT con filtro oXiris, per cercare di ottenere il maggior numero possibile di informazioni. Prima di tutto è stata misurata la qualità dei dati e, laddove necessario, sono state applicate tecniche di data cleaning e meccanismi di miglioramento al fine di rendere il dataset risultante di qualità il più possibile accettabile. Tutto ciò per permettere in un secondo momento analisi sofisticate di Machine Learning.

Nello specifico si è cercato di clusterizzare i pazienti sulla base delle caratteristiche cliniche registrate, focalizzando le analisi su molteplici outcome di maggior interesse ai fini medici.


\section{Organizzazione del testo}

Il lavoro è suddiviso nel seguente modo.\\ 
Nel capitolo 2 viene mostrato nel dettaglio il registro ARRT. Nello specifico viene descritto come si suddivide al suo interno e per ogni sezione vengono mostrati le caratteristiche più interessanti.

Nel terzo capitolo si trovano i concetti alla base della Data Quality visti sia da un punto di vista generico che da un punto di vista medico. \\
Sempre in questo capitolo viene presentato il dataset ottenuto dal registro e vengono, inoltre, illustrate e discusse le principali criticità  riscontrate analizzando le soluzioni applicate per assicurarne una buona qualità finale.

Il capitolo 4, introduce le principali tecniche di Machine Learning utilizzate in questo lavoro e mostra i risultati sperimentali ottenuti applicando tali tecniche sul dataset.

Per concludere, nel capitolo 5 si fa un'analisi finale di tutto il lavoro svolto descrivendo dei possibili sviluppi futuri.

