\chapter{Conclusioni}


Nel presente lavoro di tesi è stato presentata una tecnica di realizzazione di un dataset per la clusterizzazione di pazienti critici sottoposti a trattamenti extracorporei nei reparti di terapia intesiva. 

Come descritto nell'introduzione nella medicina basata sull'evidenza la scelta di un determinato trattamento a discapito di altri non ha delle basi solide su cui basarsi,  perchè fare ricerche in merito conduce quasi sempre a studi fallimentari e con dei costi molto alti.
Negli ultimi anni si stanno sfruttando i registri medici come strumento di analisi a posteriori sui dati dei pazienti clinici, per permettere trial più efficienti ed economici. 
Pertanto un primo passo molto importante è quello di riuscire a validare i risultati fino ad oggi ottenuti sui trial con i dati ricavati dai registri.

In base a quanto ottenuto in tutto questo lavoro i risultati si dimostrano altamente promettenti. Infatti dal punto di vista medico gli esiti ottenuti già solo con la tecnica della PCA applicata sui dati del sofa score e della mortalità sembrano, ad una prima analisi, essere in linea con gli studi avvenuti negli ultimi quarant'anni. 



\section{Considerazioni finali e sviluppi futuri}

Ovviamente le più grandi criticità riscontrate sono state la grandissima eterogeneità e la mancanza corposa dei dati.
Spesso la causa di questo si può attribuire ai diversi protocolli eseguiti tra centro e centro, ma anche una scarsa educazione sulla compilazione corretta del registro.

Proprio perché ancora un pratica in sviluppo, probabilmente ancora non è chiaro il potenziale di tale sistema,infatti se le analisi mediche sui dati clusterizzati tramite K-mean validassero ulteriormente i risultati, sarebbe possibile in futuro fondere ancor meglio la data science e la medicina permettendo eventulmente misurazioni ancora più sofisticate e utili al benessere comune.


